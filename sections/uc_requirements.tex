\section{Metadata Usecases}\label{sec:reqs}

This section describes use cases for metadata within the SKA.
The sources for these metadata include a "literature review of available SKA documents" and a survey/interview with the leaders for each of the demonstrator cases. 

\subsection{SRCNet Use Cases}

The SRCNet has already defined a number of use cases (UCs) which are described as "tasks an astronomer as well as SKAO/SRC Staff would be likely to perform within the
SRCNet" \footnote{tasks an astronomer as well as SKAO/SRC Staff would be likely to perform within the SRCNet}. 
These UCs span a wide variety of topics, not all of which concern the use of metadata.
The following UCs are all of those contained within the UC document which will have implications for the metadata landscape of the SKA. 
\\
\begin{itemize}[label={}]
    \item {\bf UC\_SRC\_1} - To browse the entire SKA data collection so that I can find data products which are suitable for my science goals.
    \item {\bf UC\_SRC\_2} - To inspect (on the fly) and retrieve data products available at a certain coordinate or for a long list of positions
    \item {\bf UC\_SRC\_3} - To make my data products and associated workflows public to maximise scientific return and avoid duplication of effort
    \item {\bf UC\_SRC\_4} - To browse the QA information associated with ODPs
    \item {\bf UC\_SRC\_5} - To track the workflow history of a data product so that I can understand how it was generated by the Observatory and/or users
    \item {\bf UC\_SRC\_6} - To optionally transfer a dataset that fails QA to the SRCNet so that I can investigate the cause of the failure
    \item {\bf UC\_SRC\_7} - To Inspect metadata, provenance, and available workflows associated with user-generated ADPs to assess their robustness
    \item {\bf UC\_SRC\_8} - To monitor usage at the project level to ensure it is in line with appropriate policies, and take appropriate action if not
    \item {\bf UC\_SRC\_9} - To track my processing time and data storage so that I can manage my compute resources
    \item {\bf UC\_SRC\_10} - To estimate the data size and processing time needed for my project before submitting a processing proposal
    \item {\bf UC\_SRC\_11} - To monitor total SRCNet account usage per user and per user group, aggregated across all SRCNet resources, currently and historically, so that I can check if a user, user group or project is still within its allocated usage and take appropriate action as required
    \item {\bf UC\_SRC\_12} - To monitor pledged and used resources per SRC site, allocated SRC project and nation, to track usage (e.g. quarterly, yearly and all historical data), so that they may be used for future plans accordingly and shared with colleagues outside the SRCNet ecosystem
    \item {\bf UC\_SRC\_13} - To monitor the progress of my processing jobs (e.g. status, used resources) so that I can manage the data reduction and identify any issues with the processing.
    \item {\bf UC\_SRC\_14} - To Monitor and understand the popularity of data products or data collections so that I can ensure that appropriate overall QoS is being provided to popular (or unpopular) data products/collections
\end{itemize}

 

Add a short description on the specifics on the metadata needed, group by the required information and include all UCs which require it. 
Collate this information in a matrix later on. 


\subsection{Demonstrator Cases of the SKA - survey}

\subsubsection{From Discussions}
\begin{itemize}[label={}]
    \item {\bf UC 1} - Storage vs Re-computation. The determination of whether a piece of data should be stored or re-computed/re-downloaded. Requires information on the size of the data, time taken to compute/download, number of times it has been accessed, available resources (compute and storage).
    \item {\bf UC 2} - Prediction of Pipeline Performance. The prediction of required resources (e.g. compute, memory, time, quality, etc.) to execute a pipeline, based upon the pipeline components and input data scale. Requires identifiers for the different pipeline components (fucntions, modules, etc.), metadata on memory consumption and time for individual components, versioning information, data source metadata for scaling, quality metrics. 
    \item {\bf UC 3} - Recommendation of Components. Requires description on the purpose of components, data flow to recommend next steps based upon previous, parameter key/values if recommendation on that level is required, data source metadata to recommend steps with appropriate scaling etc., quality metrics.  
    \item {\bf UC 4} - Retrieve all Data Products from a given Source. Data flow, data source metadata, data product metadata, identifiers on all aforementioned metadata. 
    \item {\bf UC 5} - Anomaly detection. Determination of: the difference between two pipeline executions; the source of a fault within a pipeline execution; or unusual processing/parameter values within a pipeline execution.
    \item {\bf UC 6} - Re-computation. The ability to re-execute a pipeline from the provenance.
    \item {\bf UC 7} - Retrieve all data products from person/organisation.
\end{itemize}

