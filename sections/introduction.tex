\section{Introduction}\label{sec:intro}

KR: The initial draft of the vision document of the current practice and proposals for modifications. In the future PI can be presented to and agreed by UKSRC and SRCNet architecture teams.


Metadata is data that describes other data - in general it facilitates finding and working with particular instances of the data. This document examines some existing approaches to metadata design that have been used in the general astronomy domain, with a view to making recommendations for the metadata data model to be used within SRCNet.

Metadata is a crucial building block in making a FAIR system;
\begin{itemize}
    \item \textbf{Findable} - the metadata model enables an astronomer to understand the relationships between different parts of the data at to be able to build queries.
    \item \textbf{Accessible} - Common metadata models can facilitate the creation of clients that can understand how to work with diverse datasets.
    \item \textbf{Interoperable} - Metadata models can allow mappings to be made between different datasets.
    \item \textbf{Reusable} - Metadata models can assist in being sure that code that operates on data will behave in the same way in different operating environments. In particular it can aid in understanding the provenance of a dataset and the preceding operations that have been performed on it. 
    
\end{itemize}

It should be noted that it is sometimes desirable to have different metadata models for the same data set that are customized to the particular use case. For instance it is often desirable to have a "simpler" metadata model when doing data discovery than when doing archive ingestion management for instance - this is because the typical archive user will not need to concern themselves with the same level of detail that the archive curator will. In this case it often makes implementation easier if the "data discovery" model is a sub-set or a view of the "archive management" data model, but it does not have to be the case.

\subsection{Query Languages}

One of the primary motivations behind using a metadata model is to have a concrete domain against which queries can be expressed and executed so that exact results can be obtained - To make the data findable.

\subsubsection{relational query languages}
It is common for metadata models to be expressed in a form that can be stored in relational databases. The commonest relational query language is SQL with ADQL being a variant of SQL that is customized to astronomical use cases.

