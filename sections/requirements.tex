\section{Requirements}

\subsection{Demonstrator Cases of the SKA - proposal}

A proposal document (add link when/if available) for the SKA outlined 8 initial demonstrator cases (DCs) for the SKA. 
These DCs each describe the scientists intended approach to process data from the SKA in order to complete their scientific goals. 
Within the document, the scientific outline, community involvement, objectives, timelines and requirements are described for each DC.
The requirements imposed by all DCs on the computational infrastructure have been collated and presented within Table \ref{tab:DC_req}. 

\newcommand{\man}[1][]{\(\CIRCLE_{#1}\)}     % mandatory
\newcommand{\opt}[1][]{\(\LEFTcircle_{#1}\)} % optional
\newcommand{\non}{\Circle}                   % not needed

\begin{tabular}{cc|c|c|c|c|c|c|c|c}

                            &                                   & \rot{DC 1} & \rot{DC 2} & \rot{DC 3} & \rot{DC 4} & \rot{DC 5} & \rot{DC 6} & \rot{DC 7} & \rot{DC 8} \\
\hline
\multirow{5}{*}{Platform}
                            & Python-pipelines
  & \non & \man & \non & \non & \man & \non & \non & \non \\
                            & Jupyter notebooks
  & \non & \man & \non & \non & \man & \non & \man & \non \\
                            & Interaction with ADPs
  & \non & \non & \man & \non & \non & \non & \non & \non \\
                            & Visualisation tool
  & \non & \non & \non & \non & \man & \non & \non & \non\\
                            & Testing \& Analysis
  & \non & \non & \non & \non & \non & \man & \non & \man \\

\hline
\multirow{7}{*}{Compute}
                            & Download
  & \man & \non & \non & \non & \non & \non & \non & \non \\
                            & Core hours
  & \man & \non & \non & \non & \non & \non & \non & \non \\
                            & Memory
  & \man & \man & \non & \non & \non & \non & \non & \non \\
                            & Distributed Processing
  & \man & \man & \non & \man & \non & \man & \man & \non \\
                            & Rapid imaging pipeline?
  & \non & \non & \man & \non & \non & \non & \non & \non \\
                            & source finding - large cubes
  & \non & \non & \non & \man & \non & \man & \non & \non \\
\hline
\multirow{2}{*}{Data Access} 
                            & Download external archives
  & \man & \non & \man & \man & \non & \man & \man & \man \\
                            & Store other data
  & \man & \non & \man & \man & \non & \man & \man & \man \\
                              & "Rapid" access to data products
  & \non & \non & \man & \non & \non & \non & \non & \non\\
                              & Large and "Rapid" DB for candidate viewing
  & \non & \non & \non & \non & \man & \non & \non & \non\\
                              & High sensitivity imaging access
  & \non & \non & \non & \non & \non & \non & \non & \man\\
% \hline
% \multirow{2}{*}{Metadata} & same Pipeline
%   & \non & \non & \non & \non & \non & \non & \non & \non \\
%                             & other Pipelines
%   & \non & \non & \non & \non & \non & \non & \non & \non\\

%\hline
%\multirow{2}{*}{Ext. Information}
%                            & Purpose of Call
%  & \non & \non & \non & \non & \opt & \non & \non & \non & \non \\
\end{tabular}
\label{tab:DC_req}

Of these stated requirements, those contained within the category of Data Access will have implications for the metadata landscape.
Downloading external archives is often cross referencing the same objects observed by the SKA with other instruments/ at different wavelengths, requiring metadata on the data source (such as position, etc.) that are compatible with that from other instruments.
Downloading and storing data also 

\subsubsection{Data Source Metadata}

Data source metadata here refers to metadata surrounding the SDPs, received by the scientists. 

% UC\_SRC\_1, UC\_SRC\_2, UC\_SRC\_4, UC\_SRC\_5, UC\_SRC\_6, UC\_SRC\_7, UC\_SRC\_9, UC\_SRC\_10, UC\_SRC\_11, UC\_SRC\_14, DC\_1, DC\_3, DC\_4 DC\_4, DC\_5, DC\_6, DC\_7, DC\_8.
% list is clunky, remove when full set of UCs are available and turn into table.
% is it necessary to add examples of how these information will be used?

\subsubsection{Intermediate Data Metadata}



\subsubsection{Data Product Metadata}

\subsubsection{Data Processing Metadata}

\subsubsection{Data Flow}

\subsubsection{Runtime Environment}

\subsubsection{User Metadata}
